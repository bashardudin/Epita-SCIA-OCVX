\documentclass[11pt, a4paper]{article}

\usepackage[french]{babel}
\usepackage{fancyhdr}
\usepackage[margin=.8in]{geometry}

\usepackage{Style/TeXingStyle}

\pagestyle{fancy}
\renewcommand{\headrulewidth}{1.5pt}
\renewcommand{\footrulewidth}{0.5pt}
\fancyhead[L]{EPITA\_ING2\_2020\_S8}
\fancyhead[R]{Majeures SCIA \& IMAGE}
\fancyhead[C]{OCVX}
\fancyfoot[C]{\thepage}
\fancyfoot[L]{2019}
\fancyfoot[R]{\footnotesize{\textbf{Chargés de cours :} \textsc{B.~DUDIN} \& \textsc{G.~TOCHON}}}

\pretitle{\vspace{-2.5\baselineskip} \begin{center}}
\title{%
  { \huge Un peu de calcul différentiel}%
}
\posttitle{
\end{center}
\rule{\textwidth}{1.5pt}
\vspace{-3\baselineskip}
}
\author{}
\date{}

\pdfinfo{
   /Author (Bashar Dudin)
   /Title  (Calcul Différentiel - 2019)
   /Subject (SCIA/IMAGE - Optimisation convexe)
}

\setlength\parindent{0pt}

\begin{document}

\maketitle\thispagestyle{fancy}

\begin{abstract}
  On souhaite généraliser la démarche adoptée pour résoudre
  géométriquement certains programmes linéaires au cas des problèmes
  d'optimsation non-linéaires. L'objectif est de trouver des
  hyperplans d'appui aux sous-niveaux d'une fonction objectif ; ils
  indiquent une direction de recherche pour minimiser celle-ci. La
  définition de ces hyperplans d'appui s'effectue par une étude locale
  des fonctions objectifs, une étude qui généralise l'apport de la
  dérivée d'une fonction numérique à la compréhension du comportement
  de celle-ci en un point.
\end{abstract}

\tableofcontents

\section{Normes sur $\R^n$}
\label{sec:nomesRn}

Une norme sur un espace vectoriel apporte une manière de mesurer la
\textit{longueur} d'un vecteur tout en respectant un minimum la
structure vectorielle. Elle permet en particulier de définir une
notion de distance entre deux point de $\R^n$ par la \textit{longueur}
du vecteur qui les relie. Pouvoir changer de mesure de longueur
suivant les problèmes qu'on attaque est crucial lorsque l'on s'attaque
à des problèmes d'optimisation. À la fois pour pouvoir modéliser les
problèmes en jeu ; comment mesure la différence entre deux mots? Ou
pour accélérer la convergence de certains algorithmes
d'apprentissages.

\begin{defn}
  Une norme sur $\R^n$ est une application $\|\cdot\| : \R^n \to \R_+$
  telle que:
  \begin{enumerate}
  \item $\|x\| = 0 \Leftrightarrow x = 0$ ;
  \item $\forall \lambda \in \R$, $\forall x \in \R^n$,
    $\|\lambda x\| = |\lambda|\|x\|$ (relation d'\textit{homogéinité});
  \item $\forall x \in \R^n$, $\forall y \in \R^n$,
    $\|x + y\| \leq \|x\| + \|y\|$ (\textit{Inégalité triangulaire}).
  \end{enumerate}
\end{defn}
\begin{question}
  Interpréter chacune des propriétés suivantes avec vos propres mots.
\end{question}
L'inégalité triangulaire donne lieu à une autre inégalité, appelée
\textit{inégalité triangulaire inversée} qui peut parfois être
utile\footnote{Il est très probable qu'elle ne fasse que partie de
  votre culture ...}:
\[
  \forall x, y \in \R^n, \qquad \big| \|x\| -\|y\| \big| \leq \|x -
  y\|.
\]
\begin{question}
  Représenter cette inégalité géométriquement. Essayer de la déduire
  de l'inégalité triangulaire.
\end{question}
Les trois normes les plus fréquentes d'utilisation sont les trois
suivantes, elles sont respectivement qualifiées de normes $1$, $2$ et
infinie.
\begin{enumerate}
\item $\forall x \in \R^n$, $\| x \|_1 = \sum_{i=1}^n |x_i|$ ;
\item $\forall x \in \R^n$,
  $\| x \|_2 = \sqrt{\sum_{i=1}^n x_i^2} = \sqrt{x^Tx}$ ;
\item $\forall x \in \R^n$, $\|x \|_{\infty}$,
  $\max\{|x_i| 1 \leq i \leq n\}$.
\end{enumerate}
Les définitions des normes $1$ et $2$ vont pouvoir se généraliser pour
tout $p \geq 1$\footnote{Il est tout à fait naturel de se poser la
  question de savoir pourquoi $p \geq 1$! :)}, par
\[
  \forall x \in \R^n, \qquad \|x \|_p = \left(\sum |x_i|^p
  \right)^{\frac{1}{p}}.
\]
La norme $\|\cdot\|_p$ est communément appelée la norme $p$. Montrer
le fait que c'est une norme est uniquement délicat pour ce qui est de
l'inégalité triangulaire, la preuve de ce fait se base sur des
inégalités de convexités dites de \textsc{Hölder}, on n'abordera pas
cette preuve dans ce cours.Les normes $p$ ne seront que très
localement utilisées en dehors des cas standards des normes $1$, $2$
et $+\infty$, il est cela dit usuel d'en connaître la définition.

À partir d'une norme on va être en mesure de définir:
\paragraph{Une notion de \textit{distance} entre deux points de
  $\R^n$} par norme du vecteur de qui relie les deux points en
question. Plus formellement étant donné une norme $\|\cdot\|$ on note
\[
  \forall x, y, \qquad d_{\|\cdot\|}(x, y) = \| x - y \|.
\]
Dans le cas des normes $p$ on se contente d'indiquer $p$ en indice.
\begin{question}
  Représenter graphiquement les distances $1$, $2$ et $+\infty$
  entre deux vecteurs de $\R^2$.
\end{question}
\paragraph{Une notion de \textit{voisinage} d'un point de $\R^n$} au
sens de la norme utilisée. Cette notion éminemment liée à la
première, on l'isole ici parce qu'elle apporte un point de vue
auquel vous n'avez pas encore été confrontés. Étant donné un nombre
réel $\varepsilon > 0$ on va qualifier $\varepsilon$-voisinage d'un
point $x \in \R^n$ tous les points à distance (au sens de la norme
utilisée) au plus $\varepsilon$ de $x$. Dans le jargon, on appelle
boule ouverte de rayon $\varepsilon$ et centrée en $x$ cette notion,
pour une norme ambiante $\|\cdot\|$ on note
\[
  B_{\|\cdot\|}(x, \varepsilon) = \{y \in \R^n \mid d_{\|\cdot\|}(x, y) <
  \varepsilon\}
\]
La boule fermée centrée en $x$ et de rayon $\varepsilon$ est la
notion correspondante avec les points à distance $\varepsilon$
incluses,
\[
  \overline{B}_{\|\cdot\|}(x, \varepsilon) = \{y \in \R^n \mid d_{\|\cdot\|}(x,
  y) \leq \varepsilon\}.
\]
Dans le cas des normes $p$ on se contente d'indexer les boules par
$p$.

La notion de $\varepsilon$-voisinage permet d'exprimer des
phénomènes de passage à la limite et d'études locales. Dire qu'une
suite $(u_k)_{k \in \N}$ de points de $\R^n$ converge vers
$\ell \in \R^n$, au sens d'une norme $\|\cdot\|$, correspond au fait
de dire que pour tout $\varepsilon$-voisinage $B(\ell, \varepsilon)$
de $\ell$, il existe un rang $N$ à partir duquel tous les éléments
de la suite $u_k$ sont dans $B(\ell, \varepsilon)$.
\begin{question}
  \begin{itemize}
  \item Dessiner les boules unités
    $\overline{B}_p(\underline{0}, 1)$ pour
    $p \in \{1, 2, \infty\}$.
  \item Montrer que les $\varepsilon$-voisinages d'une norme
    $\|\cdot\|$ sur $\R^n$ sont convexes.
  \item La définition d'un équivalent à une norme $p$ pour $p < 1$
    définit-il une norme?
  \end{itemize}
\end{question}
On reprend plus en détails quelques extensions des notions liées aux
études locales de fonctions en suites dans $\R$.
\begin{exmp}[Convergence d'une suite.]
  Une suite $(u_k)_{k\in \N}$ dans $\R^n$ converge vers un point
  $\ell \in \R^n$, au sens d'une norme $\|\cdot\|$ si
  \[
    \forall \varepsilon > 0, \exists N \in \N, \quad k \geq N
    \Rightarrow \|u_k - \ell \| < \varepsilon.
  \]
  Cette défintion se traduit telle quelle par:
  \[
    \forall \varepsilon > 0, \exists N \in N \quad k \geq N
    \Rightarrow u_k \in B(\ell, \varepsilon).
  \]
  Ou encore: pour tout $\varepsilon > 0$ il existe un rang $N$ à
  partir duquel tous les éléments de la suite $u_k$ sont dans le
  $\varepsilon$-voisinage de $\ell$ pour la norme $\|\cdot\|$.

  Dans le cas de la norme infinie, la définition précédente s'écrit
  explicitement comme
  \[
    \forall \varepsilon > 0, \exists N \in \N, \quad k \geq N
    \Rightarrow \max_{1 \leq i \leq n}\|u_{k, i} - \ell_i \| < \varepsilon.
  \]
  où $u_{k, i}$ (resp. $\ell$) est la composante le long de la
  coordonnée $i$ du vecteur $u_k$ (resp. $\ell_i$). Cette définition
  peut encore être réécrite
  \[
    \forall \varepsilon > 0, \exists N \in \N, \quad k \geq N
    \Rightarrow \forall i \in \{1, \ldots, n\}, \|u_{k, i} - \ell_i \| <
    \varepsilon.
  \]
  Un peu de réflexion permet de voir que cette dernière est
  équivalente
    \[
      \forall i \in \{1, \ldots, n\}, \forall \varepsilon > 0, \exists
      N \in \N, \quad k \geq N \Rightarrow \|u_{k, i} - \ell_i \| <
      \varepsilon.
  \]
  Autrement dit, la suite $(u_k)$ converge vers $\ell$, au sens de la
  norme infinie, si et seulement si, chacune des suites
  \textbf{numériques} $(u_{k, i})$ converge vers $\ell_i$.
\end{exmp}
\begin{exmp}[Limite d'une fonction en un point.]
  On considère deux normes $\|\cdot\|_\alpha$ et $\|\cdot\|_\beta$
  respectivement sur $\R^n$ et $\R^m$. Une fonction
  $f : (\R^n, \|\cdot\|_\alpha) \to (\R^m, \|\cdot\|_\beta)$ a une
  limite $\ell \in \R^m$ en un point $a \in \R^n$ si
  \[
    \forall \varepsilon > 0, \exists \eta > 0, \quad x \neq a
    \mathrm{ et } \|x - a\|_\alpha \eta \Rightarrow \|f(x) -
    \ell\|_\beta < \varepsilon.
  \]
  Chose qu'on peut encore exprimer par
  \[
    \forall \varepsilon > 0, \exists \eta > 0, \quad x \in
    B_{\|\cdot\|_\alpha}(a, \eta)\setminus \{a\} \Rightarrow f(x) \in
    B_{\|\cdot\|_\beta}(\ell, \varepsilon),
  \]
  ou encore pour tout $\varepsilon$-voisinage $B$ de $\ell$ pour la
  norme $\|\cdot\|_\alpha$ il existe un $\eta$-voisinage de $a$, a
  étant exclu dont tout élément a une image dans $B$.
\end{exmp}
\begin{exmp}[Continuité d'une fonction en un point.]
  La continuité d'une fonction
  $f : (\R^n, \|\cdot\|_\alpha) \to (\R^m, \|\cdot\|_\beta)$ en un
  point $a \in \R^n$ s'écrit
  \[
    \forall \varepsilon > 0, \exists \eta > 0, \quad \|x - a\|_\alpha
    \eta \Rightarrow \|f(x) - f(a)\|_\beta < \varepsilon.
  \]
  Autrement dit pour tout $\varepsilon$-voisinage $B$ de $f(a)$ pour
  la norme $\|\cdot\|_\alpha$ il existe un $\eta$-voisinage de $a$
  dont tout élément a une image dans $B$.
\end{exmp}
\begin{question}
  \begin{itemize}
  \item Donner des exemples de fonctions continues (pour les normes de
    votre choix) de $\R^2$ dans $\R$.
  \item Comment peut-on en construire d'autres à partir de celles-ci?
    Que dire de l'addition, la multiplication, le quotient ou la
    composition de fonctions continues?
  \item Comment tester la continuité de fonctions à plusieurs
    variables en se ramenant au cas des fonctions numériques?
  \item Est-ce que \c{c}a marche tous le temps?
  \end{itemize}
\end{question}
\begin{exmp}[Comparaisons en un point.]
  Une fonction
  $f : (\R^n, \|\cdot\|_\alpha) \to (\R^m, \|\cdot\|_\beta)$ est un
  $o$ en un point $a \in \R^n$ d'une fonction $g$, sur les mêmes
  espaces et par rapport aux mêmes normes, s'il existe une fonction
  $\epsilon : \R^n \to \R$ telle que $f = \epsilon g$ avec
  $\lim_{t \to a }\epsilon(t) = 0$. Quand $g$ n'est pas nulle sur un
  voisinage de $a$ la définition précédente est équivalente au fait que
  \[
    \lim_{t \to a} \frac{\|f(t)\|_\alpha}{\|g(t)\|_\beta}  = 0.
  \]
  On écrit dans ce cas que $f$ est un $o_a(g)$. On laisse tomber le
  point $a$ quand celui-ci est clair du contexte.

  Pour obtenir les notions d'équivalence et de $\mc{O}$ en un point
  $a$, il suffit de remplacer les limites de $\epsilon$ ou de quotient
  respectivement par $1$ et une constante positive quelconque.
\end{exmp}
\begin{question}
  Justifier les affirmations suivantes:
  \begin{enumerate}
  \item $\langle h, h \rangle$ est un $o_{\underline{0}}(h)$;
  \item
    $\sin\big(\langle h, h \rangle\big) \sim_{\underline{0}} \langle
    h, h \rangle$.
  \end{enumerate}
\end{question}
À ce stade on est en droit de se demander si le choix des normes avec
lesquels on travaille a une incidence sur les limites des suites qu'on
étudie (une même suite convergerait pour une norme et pas pour une
autre), les fonctions qu'on regarde (une même fonction serait continue
pour une norme et pas pour une autre) etc. Le théorème suivant permet
de se rassurer sur ce point (du moins en dimension finie).
\begin{defn}
  Deux normes $\|\cdot\|_\alpha$ et $\|\cdot\|_\beta$ sur $\R^n$ sont
  dites \emph{équivalentes} s'il existe des constants $c$,
  $C \in \R_+^*$ telles que
  \[
    \forall x \in \R^n, \qquad c\|x\|_\alpha \leq \|x\|_\beta \leq
    C\|x\|_\alpha.
  \]
\end{defn}
En reprenant les exemples précédents, il est relativement simple de se
convaincre que deux normes équivalentes donnent des suites de mêmes
natures, les mêmes fonctions continues et les mêmes comparaisons.
\begin{question}
  Montrer que les normes $1$, $2$ et $\infty$ sont équivalentes.
\end{question}
\begin{thm}
  Toutes les normes sur $\R^n$ sont équivalentes.
\end{thm}
\begin{rem}
  La preuve de ce théorème est en dehors du périmètre de ce
  cours. Elle nécessite certaines notions de topologie qui vous
  manquent ; notamment la notion de compacité. On se contentera du
  résultat.
\end{rem}
Le fait que les normes soient équivalentes en terme de questionnement
topologique (les études des phénomènes locaux) ne signifie pas
qu'utiliser l'une ou l'autre en modélisation revient au même.
\begin{question}
  Quelles normes utiliseriez-vous pour modéliser les problématiques
  suivantes:
  \begin{enumerate}
  \item Le calcul de la distance que doit parcourir un oiseau entre
    deux coordonnées GPS pas trop éloignées.
  \item Le calcul de la distance que parcours un touriste à Manhattan
    entre deux musée.
  \item Le calcul du nombre de modifications nécessaires (lettre à
    lettre) pour changer un mot en un autre.
  \end{enumerate}
\end{question}

\section{Différentiabilité et différentielle en un point}
\label{sec:diffetdiff}

\section{Gradient en un point}
\label{sec:gradient}

\section{Caractérisation du premier ordre de la convexité}
\label{sec:caract1convexite}

\section{Hessienne en un point}
\label{sec:hessienne}

\section{Caractérisation du second ordre de la convexité}
\label{sec:caract2convexite}


\end{document}
%%% Local Variables:
%%% mode: latex
%%% TeX-master: t
%%% End:
