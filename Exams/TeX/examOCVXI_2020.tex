\documentclass[11pt, a4paper]{article}

\usepackage[french]{babel}
\usepackage{fancyhdr}
\usepackage[margin=.8in]{geometry}

\usepackage{Style/TeXingStyle}

\pagestyle{fancy}
\renewcommand{\headrulewidth}{1.5pt}
\renewcommand{\footrulewidth}{0.5pt}
\fancyhead[L]{EPITA\_ING2\_2020\_S8}
\fancyhead[R]{Majeures SCIA \&IMAGE}
\fancyhead[C]{OCVX1}
\fancyfoot[C]{\thepage}
\fancyfoot[L]{Avril 2019}
\fancyfoot[R]{\footnotesize{\textbf{Chargés de cours :} \textsc{B.~Dudin} \& \textsc{G.~Tochon}}}

\pretitle{\vspace{-.5\baselineskip} \begin{center}}
\title{%
  { \huge Examen d'optimisation convexe}%
}
\posttitle{
\end{center}
  \begin{flushleft}
    \vspace{3\baselineskip} \textit{
      \!\!\emph{Durée de l'épreuve 3h.}\\
      \! \emph{Les documents du cours ainsi que les calculettes ne
        sont pas
        autorisées.}  \\
      Le barême est indicatif et peut évoluer de manière marginale. Il
      contient un maximum de $37$ points, les notes seront rapportées
      à une note inférieure, très probablement $\geq 20$.  }
  \end{flushleft}
  \rule{\textwidth}{1.5pt}
  \vspace{-5\baselineskip}
}
\author{}
\date{}

\pdfinfo{
   /Author (Bashar Dudin)
   /Title  (Examen optimisation convexe - 2020)
   /Subject (Optimisation convexe)
}

\begin{document}

\maketitle\thispagestyle{fancy}

\section{Géométrie élémentaire dans $\R^2$}
\label{sec:geom}

\noindent On cherche dans cette section à vous faire représenter
quelques lieux géométrique de $\R^2$.

\noindent On désigne par $A$ la partie de $\R^2$ donnée par
\[
  A = \left\{ (x, y) \in \R^2 \; \left| \;
      \begin{matrix}
        x + 2y & \leq 4 \\
        2x - 3y & \leq 2 \\
        -x + 5y & \leq 6
    \end{matrix}\right.\right\}.
\]
\begin{question}{3}
  \begin{enumerate}
  \item
    Représenter $A$ graphiquement en indiquant les éléments qui
    permettent de décomposer votre représentation.
  \item Quelle contrainte (d'inégalité) linéaire faut-il rajouter aux
    contraintes de $A$ pour avoir un lieu borné de $\R^2$ délimité par
    une droite parallèle à $x + 2y = 4$ et passant par $(-1,-2)$? On
    note $B$ la partie ainsi obtenue.
  \item Donner l'équation d'un hyperplan d'appui à $B$ au point
    $(\frac{16}{7}, \frac{6}{7})$.
  \end{enumerate}
\end{question}
On note $C$ l'intersection de $A$ avec l'épigraphe de la fonction
$\varphi: x \mapsto -\ln(x+5)$ sur son domaine de définition.
\begin{question}{3}
  \begin{enumerate}
  \item Représenter $C$ graphiquement (schématiquement).
  \item Comment représenter le graphe de $\varphi$ comme un courbe de
    niveau?
  \item En quoi est-ce que $C$ est convexe\footnote{Autrement que par
      le fait que convexe \c{c}a commence par un $C$!}?
  \end{enumerate}
\end{question}
On considère les fonctions suivantes notées $g$ et $h$ respectivement
données par les expressions
\[
  g(x, y) = x^3- y, \quad h(x, y) = x^2-y^2.
\]
\begin{question}{2}
  \begin{enumerate}
  \item Représenter les courbes de niveaux
    $\mc{C}_0(g)$ et $\mc{C}_1(h)$.
  \item Justifier le fait que $g$ ne soit pas convexe.
  \end{enumerate}
\end{question}
\begin{question}{1}
  Justifier soigneusement le fait que toute norme de $\R^n$ définit
  une fonction convexe.
\end{question}
\begin{question}{1}
  En quoi est-ce que la condition de convexité a un intérêt dans les
  problèmes d'optimisation?
\end{question}

\section{Calcul différentiel élémentaire}

\noindent À moins qu'on y fasse explicitement référence, il n'est pas
nécessaire de justifier la différentiabilité des fonctions que vous
manipulez.
\begin{question}{2}
  Expliciter le gradient des expressions suivantes en tout point où
  cela fait sens:
  \begin{enumerate}
  \item $f(x, y, z) = xy^2 + \cos(xz) -\tan(x^2)$;
  \item $g(x, y, z) = \ln(1 + x^2 + y^2)$.
  \end{enumerate}
  Quelle est la jacobienne de la fonction définie par $h = (g, f^2)$.
\end{question}
\begin{question}{2}
  Quelle est la dérivée directionnelle en $(0, 0)$ de la fonction
  \[
    f(x, y) = \left\{
      \begin{array}{cc}
        \frac{xy}{x+ y} & \textrm{si $(x, y) \neq (0, 0)$} \\
        0 & \textrm{sinon}
      \end{array}\right.
  \]
  le long de $(1, \alpha)$ pour
  $\alpha \in \R$. Est-ce que $f$ est différentiable en $(0, 0)$?
\end{question}

\noindent On se donne une matrice $A \in \mc{M}_n(\R)$.
\begin{question}{2}
  Calculer, là où cela fait sens, la différentielle des fonctions
  suivantes:
  \begin{enumerate}
  \item $f :\R^n \rightarrow \R$ donnée par l'expressoin
    $X \mapsto X^T$.
  \item $g : \R^n \rightarrow \R$ donnée par l'expression
    $X \mapsto \tan(X^TAX)$.
  \end{enumerate}
\end{question}

\section{Problèmes d'optimisation simples}

\subsection{De l'existence de points optimaux}

\noindent On considère les contraintes linéares sur $\R^2$
\[
\systeme{
  x + 2y \leq 3,
  x - y  \geq 2
}
\]
définissant le lieu $D$ de $\R^2$.
\begin{question}{2}
  Donner pour chacune des propriétés suivantes un programme linéaire
  ayant $D$ pour lieu admissible\footnote{Lieu décrit par les
    contraintes du programme linéare.} et satisfaisant cette propriété
  \begin{enumerate}
  \item le programme linéaire n'a qu'un seul point optimal ;
  \item le programme linéaire en a une infinité;
  \item le programme linéaire n'est pas borné.
  \end{enumerate}
\end{question}

\subsection{Un programme linéaire en petite dimension}

\noindent On considère le programme linéaire (\emph{PL}) suivant :
\begin{displaymath}
  \begin{linearProg} {
      minimiser
    }{
      $x + y$
    }{
      sujet à $(x, y) \in B$
    }
  \end{linearProg}
\end{displaymath}
On cherche dans la suite à étudier (\emph{PL}).
\begin{question}{4}
  \begin{enumerate}
  \item En reprenant éventuellement la représentation graphique que
    vous avez utilisée section \ref{sec:geom}:
    \begin{enumerate}
    \item[a.]  Tracer la courbe de niveau $0$ de la fonction objectif
      de (\emph{PL}). Elle sera notée $\mc{C}_0$.
    \item[b.]  Indiquer les demi-espaces positif et négatif définis
      par $\mc{C}_0$.
    \item[c.]  Indiquer dans quelle direction on doit translater
      $\mc{C}_0$ afin de minimiser la fonction objectif
    \end{enumerate}
  \item
    Tracer la courbe de niveau qui réalise le minimum de (\emph{PL})
    et calculer l'unique point optimal de (\emph{PL}). Quelle est la
    valeur optimale de (\emph{PL})?
  \end{enumerate}
\end{question}

\subsection{\emph{Baby examples}}

\noindent L'approche géométrique dans le cas des programmes linéaires
de petites dimensions s'étend à certains problèmes d'optimisations
simples. On considère le probème d'optimisation (\emph{P1}) suivant
\[
\begin{PbOptim}{
    minimiser
  }{
    $2x + y$
  }{
    $(x, y) \in C$
  }
\end{PbOptim}
\]
où $C$ est la partie de $\R^2$ qu'on a décrite section \ref{sec:geom}.

\begin{question}{3}
  \begin{enumerate}
  \item Représenter la courbe de niveau de la fonction objectif qui
    réalise le minimum de (\emph{P1})\footnote{croquis de son
      positionnement.}.
  \item Calculer le point optimal ainsi que la valeur optimale de
    (\emph{P1}).
  \end{enumerate}
\end{question}
On inverse nos habitudes dans la suite pour étudier un problème
d'optimisation (\emph{P2}) où les contraintes sont plus simples que la
fonction objectif.
\[
\begin{PbOptim}{
    minimiser
  }{
    $x^2 + y^2$
  }{
    $2x + 3y \geq 6$
  }
\end{PbOptim}
\]
\begin{question}{3}
  Trouver le point minimal de ce problème d'optimisation. Expliquer
  votre démarche.
\end{question}

\section{Problèmes d'optimisation plus construits}
\label{sec:plusdur}

\noindent On considère le problème d'optimisation (\emph{P3}) suivant
\[
\begin{PbOptim}{
    minimiser
  }{
    $xy + x^2 + y^2 + z^2$
  }{
    $y + z \leq -1$
    }
\end{PbOptim}
\]
\begin{question}{5}
  \begin{enumerate}
  \item Écrire (\emph{P3}) sous forme standard.
  \item Justifier le fait que (\emph{P3}) soit un problème
    d'optimisation convexe.
  \item Donner le Lagrangien de (\emph{P3}) puis son problème dual
    (\emph{D3}).
  \item Résoudre (\emph{D3}).
  \item En déduire une solution du problème primal (\emph{P3}).
  \end{enumerate}
\end{question}

\noindent On considère le problème d'optimisation (\emph{P4})
\[
\begin{PbOptim}{
    minimiser
  }{
    $\alpha x + \beta y$
  }{
    \systeme{
      x^2 + y^2 \leq 5,
      x^2 + 5y^2 \geq 9
    }
  }
\end{PbOptim}
\]
On s'intéresse aux solutions des problèmes d'optimisation pour
$(\alpha, \beta)$ qui varient.
\begin{question}{4}
  \begin{enumerate}
  \item Calculer un point optimal quand $(\alpha, \beta) = (1, 0)$
    puis $(0, 1)$.
  \item Où se trouve le point optimal quand
    $(\alpha, \beta) = (4, 10)$?
  \item Donner le point d'intersection des bords des sous-niveaux
    définissant les contraintes de notre problème d'optimisation.
  \item Caractériser tous les paramètres $(\alpha, \beta)$ qui ont le
    point d'intersection dans le troisième cadran comme point optimal.
  \end{enumerate}
\end{question}

\end{document}

%%% Local Variables:
%%% mode: latex
%%% TeX-master: t
%%% End:
